\section{Goal of the METRICSTICS system}


Increase the accuracy and reliability of statistical calculations performed by the calculator to achieve a margin of error of less than 1\% and ensure a high level of reliability for all 7 used statistical functions by the end of November 2023.
 
\subsection*{Specific}
\addcontentsline{toc}{subsection}{\protect\numberline{}Specific}
The goal is specific about what needs improvement - the accuracy of statistical calculations. It specifies the desired outcome, which is achieving a margin of error of less than 1\%.

\subsection*{Measurable}
\addcontentsline{toc}{subsection}{\protect\numberline{}Measurable}
Accuracy can be measured by comparing the calculator's results with known correct results for various statistical functions. Achieving a margin of error of less than 1\% provides a clear, quantifiable benchmark.

\subsection*{Attainable}
\addcontentsline{toc}{subsection}{\protect\numberline{}Attainable}
Improving accuracy in statistical calculations is achievable through rigorous testing, validation, and refinement of the underlying algorithms and code in the calculator.

\subsection*{Realistic}
\addcontentsline{toc}{subsection}{\protect\numberline{}Realistic}
Accuracy is a critical factor for a statistics calculator. Users rely on it for precise calculations in fields such as research, data analysis, and decision-making.

\subsection*{Timely}
\addcontentsline{toc}{subsection}{\protect\numberline{}Timely}
Achieve a margin of error of less than 1\% within the next 2 months.
\hfill \break
\hfill \break


\section{Questions with applied metrics}

\subsection*{Question 1}
\addcontentsline{toc}{subsection}{\protect\numberline{}Question 1}
How can we improve the accuracy of METRICSTICS in calculating  minimum (m) and maximum (M) values for different data sets?
\begin{itemize}
    \item Metric 1: Mean Absolute Error (MAE) between METRICSTICS' calculated minimum (m) and actual minimum.
    \item Metric 2: Mean Absolute Error (MAE) between METRICSTICS' calculated maximum (M) and actual maximum.
\end{itemize}

\subsection*{Question 2}
\addcontentsline{toc}{subsection}{\protect\numberline{}Question 2}
How can we improve the correctness of METRICSTICS in determining mode (o) for data sets with large sample sizes?
\begin{itemize}
    \item Metric 1: Mode calculation time for different data set sizes.
    \item Metric 2: Percentage of correctly identified modes for data sets with varying degrees of multimodality.
\end{itemize}

\subsection*{Question 3}
\addcontentsline{toc}{subsection}{\protect\numberline{}Question 3}
What strategies can be implemented to effectively handle even and odd sample sizes when calculating the median (d)?
\begin{itemize}
    \item Metric 1: Median calculation time for data sets with odd and even sample sizes
    \item Metric 2: Accuracy of METRICSTICS in calculating the median compared to established methods for different sample sizes.
\end{itemize}

\subsection*{Question 4}
\addcontentsline{toc}{subsection}{\protect\numberline{}Question 4}
How can we improve the accuracy of METRICSTICS in calculating the arithmetic mean ($\mu$) for data sets with outliers?
\begin{itemize}
    \item Metric 1: Mean Absolute Error (MAE) between METRICSTICS' calculated mean ($\mu$) and a robust mean calculation method (e.g., trimmed mean).
    \item Metric 2: Percentage of data sets where METRICSTICS' mean calculation is influenced by outliers.
\end{itemize}

\subsection*{Question 5}
\addcontentsline{toc}{subsection}{\protect\numberline{}Question 5}
How can we reduce the mean absolute deviation (MAD) in METRICSTICS' descriptive statistics?
\begin{itemize}
    \item Metric 1: MAD value for data sets processed by METRICSTICS compared to traditional MAD calculation.
    \item Metric 2: Percentage reduction in MAD achieved by implementing optimization techniques.
\end{itemize}

\subsection*{Question 6}
\addcontentsline{toc}{subsection}{\protect\numberline{}Question 6}
How can we improve the accuracy of METRICSTICS in calculating the standard deviation ($\sigma$) for data sets with varying distributions? distributions?
\begin{itemize}
    \item Metric 1: Mean Squared Error (MSE) between METRICSTICS' calculated standard deviation ($\sigma$) and actual standard deviation.
    \item Metric 2: Standard deviation error as a function of data set skewness and kurtosis.
\end{itemize}

\subsection*{Question 7}
\addcontentsline{toc}{subsection}{\protect\numberline{}Question 7}
What strategies can be implemented to ensure the reliability and stability of METRICSTICS' calculations?
\begin{itemize}
    \item Metric 1: Calculation stability over multiple runs with the same data.
    \item Metric 2: Frequency of software crashes or errors during calculations.
\end{itemize}

\subsection*{Question 8}
\addcontentsline{toc}{subsection}{\protect\numberline{}Question 8}
How can we ensure METRICSTICS remains adaptable to different data types and formats?
\begin{itemize}
    \item Metric 1: Number of supported data formats and data types.
    \item Metric 2: Frequency of updates and additions to accommodate new data structures.
\end{itemize}

\subsection*{Question 9}
\addcontentsline{toc}{subsection}{\protect\numberline{}Question 9}
What is the team's capacity for maintaining and improving METRICSTICS over time?
\begin{itemize}
    \item Metric 1: Team workload in terms of METRICSTICS-related tasks.
    \item Metric 2: Average time to address and implement user feedback and feature requests.
\end{itemize}


\subsection*{Question 10}
\addcontentsline{toc}{subsection}{\protect\numberline{}Question 10}
How can we enhance the precision and numerical stability of METRICSTICS for very large or very small data values?
\begin{itemize}
    \item Metric 1: Assessment of numerical precision using double precision arithmetic.
    \item Metric 2: Handling of extreme values in calculations.
    
\end{itemize}